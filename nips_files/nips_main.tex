\documentclass{article}

% if you need to pass options to natbib, use, e.g.:
%     \PassOptionsToPackage{numbers, compress}{natbib}
% before loading neurips_2020

% ready for submission
% \usepackage{neurips_2020}

% to compile a preprint version, e.g., for submission to arXiv, add add the
% [preprint] option:
%     \usepackage[preprint]{neurips_2020}

% to compile a camera-ready version, add the [final] option, e.g.:
%     \usepackage[final]{neurips_2020}

% to avoid loading the natbib package, add option nonatbib:
     \usepackage[nonatbib]{neurips_2020}

\usepackage[utf8]{inputenc} % allow utf-8 input
\usepackage[T1]{fontenc}    % use 8-bit T1 fonts
\usepackage{hyperref}       % hyperlinks
\usepackage{url}            % simple URL typesetting
\usepackage{booktabs}       % professional-quality tables
\usepackage{amsfonts}       % blackboard math symbols
\usepackage{nicefrac}       % compact symbols for 1/2, etc.
\usepackage{microtype}      % microtypography

\title{AlphaHalide: Scheduling Programs with Monte Carlo Tree Search}

% The \author macro works with any number of authors. There are two commands
% used to separate the names and addresses of multiple authors: \And and \AND.
%
% Using \And between authors leaves it to LaTeX to determine where to break the
% lines. Using \AND forces a line break at that point. So, if LaTeX puts 3 of 4
% authors names on the first line, and the last on the second line, try using
% \AND instead of \And before the third author name.

\author{%
  David S.~Hippocampus\thanks{Use footnote for providing further information
    about author (webpage, alternative address)---\emph{not} for acknowledging
    funding agencies.} \\
  Department of Computer Science\\
  Cranberry-Lemon University\\
  Pittsburgh, PA 15213 \\
  \texttt{hippo@cs.cranberry-lemon.edu} \\
  % examples of more authors
  % \And
  % Coauthor \\
  % Affiliation \\
  % Address \\
  % \texttt{email} \\
  % \AND
  % Coauthor \\
  % Affiliation \\
  % Address \\
  % \texttt{email} \\
  % \And
  % Coauthor \\
  % Affiliation \\
  % Address \\
  % \texttt{email} \\
  % \And
  % Coauthor \\
  % Affiliation \\
  % Address \\
  % \texttt{email} \\
}
\usepackage{color}

\usepackage{gensymb}
\newcommand{\defcommenter}[2]{%
  \expandafter\newcommand\csname #1\endcsname[1]{%
  {\color{#2}[#1: ##1]}%
  }%
}
\defcommenter{TODO}{red}
\begin{document}

\maketitle

\begin{abstract}
We explore applying the Monte Carlo Tree Search (MCTS) algorithm in a notoriously difficult task: tuning programs for high-performance deep learning and image processing. We build our framework on top of Halide and show that MCTS can outperform the state-of-the-art beam-search algorithm. Unlike beam search, which is guided by greedy intermediate performance comparisons between partial and less meaningful schedules, MCTS compares complete schedules and looks ahead before making any intermediate scheduling decision. We further explore modifications to the standard MCTS algorithm as well as combining real execution time measurements with the cost model. Our results show that MCTS can outperform beam search on a suite of 16 real benchmarks. 
\end{abstract}
\section{Introduction}
\TODO{Ameer: par1, define the problem of scheduling}\newline
\TODO{Ameer: par2, current solutions and their drawbacks}\newline
\TODO{Ameer: par3, describe our solution and bg about it}\newline
\TODO{Ameer: par4, results + contributions}\newline
\section{Background}
\subsection{Halide Scheduling}
\subsection{Beam Search}

\subsection{Monte Carlo Tree Search}
\subsection{Value Function Estimation
}

\TODO{Jenny: describe beam search and prior work}\newline
\TODO{Jenny: related work}\newline
\TODO{Jenny: describe MCTS in detail, mention the survey I shared you and the convergence guarantees}\newline
\TODO{Jenny: describe value function estimation}\newline
\section{The Proposed Scheduler}
\TODO{Ameer: Describe how MCTS works in the context of the proposed scheduler}\newline
%\TODO{Ameer: Describe how the value function estimation works}\newline
\TODO{Hasan: Describe how the random programs generator works and used}\newline
%\TODO{Ameer/Billy: Describe the training procedure with val function, mcts, cost model and features}\newline
%\TODO{Ameer: Describe how the inference mode works}\newline
\section{Results}
\TODO{Methodology: how we obtained the results, how and how long the training was done, the different hyperparameters}\newline
\TODO{More insights on the experiments and the hardware we run on.}\newline
\TODO{execution time comparison, between the beam search, ours, and master on the beam search paper's apps and their CPU}\newline
\TODO{execution time comparison, between the beam search, ours, and master on the beam search paper's apps and GPUS}\newline
\TODO{autotuning/training time comparison}\newline
\TODO{Ameer: some insights on learned schedules}\newline
\TODO{Hasan: Nice to Have runs on Gemmini and Hwacha}\newline
\TODO{Jenny: Nice to Have runs on imitation learning beam search}\newline
\TODO{Billy/Jenny: Nice to Have comparison against tiramisu}\newline

\section{Conclusion}
\TODO{Ameer: bla bla}
\bibliography{main}

\end{document}
